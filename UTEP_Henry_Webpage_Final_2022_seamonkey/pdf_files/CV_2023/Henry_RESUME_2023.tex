% LaTeX resume using res.cls
\documentclass[margin]{res}
%\usepackage{helvetica} % uses helvetica postscript font (download helvetica.sty)
%\usepackage{newcent}   % uses new century schoolbook postscript font 
\setlength{\textwidth}{5.3in} % set width of text portion

\begin{document}
% Center the name over the entire width of resume:
 \moveleft.5\hoffset\centerline{\large\bf Henry R. Moncada}
% Draw a horizontal line the whole width of resume:
 \moveleft\hoffset\vbox{\hrule width\resumewidth height 1pt}\smallskip
% Address begins here
% Again, the address lines must be centered over entire width of resume:
 \moveleft.5\hoffset\centerline{500 Rubin Dr, Apt 1403 }%1469 Zepol Rd, Apt 110}
 \moveleft.5\hoffset\centerline{El Paso, TX 79912}%Santa Fe, NM 87507}
 \moveleft.5\hoffset\centerline{(915) 730-2550}
 \moveleft.5\hoffset\centerline{hrmoncadalopez@utep.edu/hrmoncada@gmail.com}
% Draw a horizontal line the whole width of resume:
\moveleft\hoffset\vbox{\hrule width\resumewidth height 1pt}\smallskip
\begin{resume}


\section{OBJECTIVE}
%Aurora Solar - Software Engineer, Computational Scientist
% Lawrence Livermore National Laboratory - Software Engineer, Computational Scientist
%Virgin Galactic
%ECP - Oak Ridge National Laboratory (ORNL)
%Jobs position at Palantir
%Sandia National Laboratory
%Lawrence Livermore National Laboratory
% To be used by Oak Ridge National Laboratory (ORNL) 
% Job position at LANL
%Argonne National Laboratory
% EXXONMOBIL
% Job Position at ARM
% Job Position at AMD
% Graduate Certificate in Big Data Analytics
% Argonne National Laboratory HPC
% Oak Ridge National Laboratory (ORNL)
% Job Position at Sandia National Laboratory
% Argonne National Laboratory - (413322 RD1 Software Engineering Associate/RD2 Software Engineering Specialist)
% University of Chicago, Globus.
%QUALCOMM %Software Engineer at Company-Amazon Web services(AWS)
% L\&T Technology Services
% Apple's jobs
% Position at Microsoft
%Software Engineer - GPU Performance 
% Xanadu Quantum Technologies Inc
% Amazon
%Research Software Developer for Quantum Computing Architectures
% Job Position at Oak Ridge National Laboratory (ORNL)
% Job Position at Sandia National Laboratory
% Postdoctoral position at Argonne National Laboratory. 
% Oak Ridge National Laboratory (ORNL) %
% Intel - Math Algorithm Engineer oneMKL
% Postdoctoral Research Associate at Oak Ridge National Laboratory (ORNL)
% Oak Ridge National Laboratory (ORNL) 
% H1-B Visa application
% Research position in High Performance Computing
% LLNL - Applied Mechanics/Physics Postdoctoral Researcher
% Astrobiological Molecular Dynamics Modeling Postdoctoral Researcher
% Postdoctoral Fellow in Big Neuroscience Data of Forensic Populations - Kiehl Lab
% Post doctoral
% Software Engineer
% DXC Technologies
% for the Postdoctoral Appointee - Lawrence Livermore National Laboratory
% Computational Structural Biology and Computational Crystallography.
% Parallel Programming Models and Runtime Systems
% Post doctoral Scholar – Climate Dynamics\/Atmospheric Science
% HPC Engineer - Oak Ridge National Laboratory (ORNL)
% Postdoctoral Research in Numerical Analysis \& Scientific Computing
% Argonne National Laboratory. 
% to Argonne Training Program on Extreme-Scale Computing 
% Amazon interview
% LANL Foreign Visitor Documents Renewal %Job application% for Postdoctoral Researcher 
% \section{STATUS}
% {\sl Peruvian citizen}\\
% {\sl Student visa\textbf{ F1}}


  
\section{EDUCATION} % \sl will be bold italic in New Century Schoolbook (or any postscript font) and just slanted in Computer Modern (default) font
{\sl \bf Ph.D in Computational Science}, \\
University of Texas at El Paso, El Paso, TX, US, May 14, 2018, (GPA: 3.9) \\
\textit{``Parallelization and Scalability Analysis of the 3D Spatially Variant Lattice Algorithm''}\\
%
{\sl \bf Master of Science in Computational Science,} \\
University of Texas at El Paso, El Paso, TX, US, May 14, 2016, (GPA: 3.9) \\
\textit{``Towards the Scalability and Hybrid Parallelization of a Spatially Variant Lattice Algorithm''}\\
%
{\sl \bf Master of Science in Applied Mathematics}\\
University of New Mexico (UNM), Albuquerque, NM, US, Aug 1, 2009 (GPA: 3.83) \\
\textit{``Paratransgenic Vectors and their Potential Influence on the Dynamics of Chagas Disease''} \\    
%
{\sl\bf Bachelor of Science, Physics}\\
Universidad Nacional Mayor de San Marcos (UNMSM), Lima, Peru, March 17, 1998\\
\textit{``Study of Structural Property for Reflectivity Modeling System Superlattices''}   

\section{PERSONAL WEBPAGE}
{\sl \textbf{http://utminers.utep.edu/hrmoncadalopez/}\\
\sl \textbf{https://www.linkedin.com/in/henry-r-moncada-32412331/}
}
                 
\section{COMPUTER \\ SKILLS}
{\sl Operating Systems:} Linux, Windows%\hfill\\

{\sl Programming Languages and Libraries:} %\hfill \\
  \begin{itemize}  \itemsep -1pt 
      \item Proficient in C/C++,  FORTRAN 77/90,
      \item Proficient in Python 3/2.7
      \item Proficient Matlab/Octave 
      \item Familiar using Pytorch, TensorFlow, and Keras 
      \item Familiar with R and Maple/Maxima
      \item Familiar with MPI - CPU multicore clusters
      \item Familiar with Nivdia CUDA - Graphical processing units (GPUs)
      \item Familiar with AMD HIP - Graphical processing units (GPUs)
      \item Familiar with Intel SYCL -  Heterogeneous platforms (CPUs \& GPUs) 
      \item Familiar with OpenMP/OpenACC -  Directives for heterogeneous platforms (CPUs \& GPUs) 
      \item Familiar with Trilinos-Kokkos - Framework for heterogeneous platforms (CPUs \& GPUs)     
    \end{itemize}

{\sl Version Control Software :} %\hfill \\
  \begin{itemize}  \itemsep -1pt 
      \item GIT and Subversion (SVN) 
    \end{itemize}    

{\sl Debugging Tools:}
  \begin{itemize}  \itemsep -1pt  
      \item Valgrind
      \item Arm Forge
      \item Totalview  
    \end{itemize}
    
{\sl Parallel Profiling and Tracing for Performance Analysis:}
  \begin{itemize}  \itemsep -1pt  
      \item Lightweight Profiling Library for MPI (mpiP)
      \item Cray Perftools
      \item Tuning Analysis Utilities (TAU)
      \item Score-P, Scalasca, Vampir
      \item HPCToolkit
      \item Intel Advisor and VTune Amplifier
    \end{itemize}
    
{\sl Computational Scientific  Libraries :}
  \begin{itemize}  \itemsep -1pt  
      \item BLAS/LAPACK - Basic Linear Algebra Subprograms \& Linear Algebra PACKage 
      \item FFTW - Fastest Fourier Transform in the West
      \item CSPARSE - SuiteSparse, A Concise Sparse Matrix Package in C
      \item KOKKOS - Trilinos Core Kernels Package.
      \item METIS \& PARMETIS - Serial and parallel Graph Partitioning and Fill-reducing Matrix Ordering
      \item PETSc - Portable, Extensible Toolkit for Scientific Computation
    \end{itemize}
    
{\sl Package Manager :}
  \begin{itemize}  \itemsep -1pt  
      \item SPACK - Package manager tool for supercomputers. Designed to support multiple versions and configurations of software on a wide variety of platforms and environments.
%       \item 
%       \item 
%       \item 
%       \item 
%       \item
    \end{itemize}    
    
\section{WORK EXPERIENCE}
{\sl University of Texas at El Paso\\
500 W University Ave, El Paso, TX  79902\\
Dr. Shirley V. Moore\\
Professor, Computer Science,\\
%(915) 747-6836\\%(915)227-0329 /
\verb+svmoore@utep.edu+
} \hfill  
\begin{itemize}
\item Research Scientist Assistant (from Mar 2021 until now)
\end{itemize}

{\sl Unemployed
} \hfill 
\begin{itemize}
\item  Waiting for UTEP to complete job paperwork. (Feb 2021)
\end{itemize}

{\sl Los Alamos National Laboratory \\
Fluid Dynamics and Solid Mechanics (T-3) and
Computational Physics and Methods (CCS-2) Groups.\\
Los Alamos, NM 87545\\
Dr. Kim Orskov Rasmussen, (505)-6653851
\verb+kor@lanl.gov +
} \hfill  
\begin{itemize}
\item Postdoctoral Research Associate (from Jan 2019 to Jan 2021)
\end{itemize}

{\sl University of Texas at El Paso\\
500 W University Ave, El Paso, TX  79902\\
Dr. Ming-Ying Leung\\
Professor, Mathematical Sciences,\\
Director, Bioinformatics and Computational Science Programs,\\
Director, BBRC Bioinformatics Core,\\
(915) 747-6836\\%(915)227-0329 /
\verb+mleung@utep.edu+
} \hfill  
\begin{itemize}
\item Postdoc Volunteer (from Sep 2018 to Jan 2019)
\end{itemize}

{\sl El Paso Community College, Valle Verde\\
919 Hunter,  El Paso, TX 79915\\
Mrs. Olga Thurman\\
Manager, Academic Resources\\
(915) 831-2740\\
\verb+othurman@epcc.edu+
} \hfill  
\begin{itemize}
\item Math Tutor (from Nov to Dec 2018)
\end{itemize}

{\sl Focus Logistics Inc\\
1011-B Burgundy El Paso, TX 79907\\
Mrs. Lisa Garcia,\\
General Manager- El Paso Office\\
(915) 778-1301\\
\verb+lisa@focuslogisticsinc.com+
} \hfill  
\begin{itemize}
\item Data Analyst (Sep 2018)
\end{itemize}

{\sl Unemployed
} \hfill 
\begin{itemize}
\item Searching for a job, getting interviews and making job applications. (Jun-Aug 2018)
\end{itemize}

% {\sl Fall 2017 and Spring 2018 - Oak Ridge National Laboratory \& University of Texas at El Paso} \hfill  
% 
% {\sl Oak Ridge National Laboratory - Computational Science Institute - Future Technologies Group 
% Dr. Shirley V. Moore\\
% (865) 576-4838\
% \verb+mooresv@ornl.gov+
% } \hfill \\  
% \begin{itemize}
% \item Summer Intern (Jun to Aug 2017)
% \end{itemize}


{\sl University of Texas at El Paso\\
500 W University Ave, El Paso, TX  79968\\
Dr. Shirley V. Moore\\
Associate Professor, Computational Science,\\
(915) 747-6836\\%(915)227-0329 /
\verb+svmoore@utep.edu+
} \hfill  
\begin{itemize}
\item Research Assistant (Aug 2012 - May 2018)
\end{itemize}

{\sl Universidad Nacional del Callao - Peru\\
Facultad de Ciencias Naturales y Matemática 
} \hfill  
\begin{itemize}
\item Associate Professor (Mar 2012 - Jul 2012)
\end{itemize}

{\sl Unemployed - Peru
} \hfill 
\begin{itemize}
\item Searching for a job, getting interviews and making job applications (Jan-Feb 2012)
\end{itemize}
% %{\sl Colleges} \hfill \\
% \textbf{Instructor (Full Time)}, Universidad Nacional del Callao (UNAC), Callao-Peru, Faculty of Natural Science and Mathematics. \hfill { \it 2010 to July-2012\/} \\
% \textbf{UNM-Mathematics Department}, Teacher assistant (TA).  \hfill {\it Spring 2006 to 2009 \/}\\
% \textbf{UNM-Physics Department}, Part-time teacher assistant (PTA). \hfill {\it Fall 2007 to Spring 2008\/}\\
% \textbf{UNM-Multicultural Engineering Program}, Residential assistant (RA). \hfill {\it Summer 2007, 2008\/}\\
% \textbf{UNM-Johnson Gym}, Math-Tutor. \hfill {\it Fall 2006\/} \\
% \textbf{Instructor (Part Time)}, Universidad Nacional Mayor de San Marcos (UNMSM), Lima-Peru, Faculty of Physics \hfill {\it April-December 2002, April-July 2003\/} \\
% \textbf{Instructor (Full Time)}, Universidad Nacional del Callao (UNAC), Callao-Peru, Faculty of Natural Science and Mathematics.  \hspace{3cm}    {\it April-1998 to July-2003\/}    
%   
\section{LANGUAGES}
{\sl Spanish,} native language\\
{\sl English,} speak fluently, read and write with high proficiency

%\clearpage
% \section{SKILLS}  
% {\sl Problem Solving} \hfill \\ 
%    \begin{itemize}\itemsep -2pt
%       \item Excellent analytical and logical reasoning skills.  Able to multi-task. Can learn new skills quickly. Able to lead or
%             work within a group environment.
%   \end{itemize} 
% 
% {\sl Teaching skill} \hfill \\      
%     \begin{itemize}\itemsep -2pt
%       \item Enjoy working and helping students to increase their motivation to study math and physics 
%             and develop and improve new learning skills to solve math and physics problems.
%     \end{itemize} 

\section{RESEARCH EXPERIENCE}
{\sl Mar 2021 until Today - Research Scientist Assistant, University of Texas at El Paso} %\hfill \\      
\begin{itemize}\itemsep -2pt
\item GAMESS ECP project, GAMESS stand for General Atomic and Molecular Electronic Structure System. 
The main focus will be on combined CPU+GPU performance and scalability analysis for the ECP exascale machines. A preliminary assessment of the performance analysis requirements will be fleshed out in more detail through surveying GAMESS ECP developers for their requirements. The framework will be constructed from selected features from the available vendor and ECP Software Technology performance analysis tools, with the addition of project-specific tools, scripts, and benchmark data.
\end{itemize}

{\sl Jan 2019 to Jan 2021 - Postdoc Research Associate, Los Alamos National Laboratory} %\hfill \\      
\begin{itemize}\itemsep -2pt
\item E3SM ECP project, MPAS stands for Model for Prediction Across Scales. It is a collaborative project for developing atmosphere, ocean,  and other earth-system simulation components for climate, regional climate, and weather studies. It uses a hexagonal mesh resembling a honeycomb that can be stretched wide in some regions and compressed for higher resolution in others.
The MPAS framework code is written in Fortran and usesthe  Message Passing Interface (MPI)  standardized library (not a language) for the collection of processes communicating via message passing. Shared memory parallelization through OpenMP (an API) is also supported, but the implementation is left up to each core component.  
% MPAS code also includes code modification by using OpenACC compiler directives for the GPU functions to get better performance by using OpenACC on the GPU side, while maintaining the performance and correctness of our baseline code. The MPAS-ocean is one component within the MPAS framework of climate models. MPAS-Ocean is an unstructured-mesh ocean model capable of using enhanced horizontal resolution in selected regions of the ocean domain.
\end{itemize}

{\sl Doctoral Research Topic, Computational Science Department - UTEP} %\hfill \\      
\begin{itemize}\itemsep -2pt
\item The purpose of this research is to design a faster implementation of the spatially variant that improves its performance when it is running on a parallel computer system. 
The spatially variant is used to synthesize a spatially variant lattice for a periodic electromagnetic structure. The spatially variant has the ability to spatially vary the unit
cell orientation and exploit its directional dependencies. The spatially variant produces a lattice that is smooth, continuous and free of defects. The lattice spacing remains strikingly
uniform when the unit cell orientation, lattice spacing, fill fraction and more are spatially varied. This is important for maintaining consistent properties throughout the lattice.
Periodic structures like a photonic crystal or metamaterial devices can be enhanced using the spatially variant to unlock new physics applications.
Our current effort is to write a portable spatially variant code for parallel architectures. To develop and write the code, we pick a general-purpose programming language
that supports structured programming. For the parallel code, we use FFTW for handling the Fourier Transform of the unit cell device and PETSc 
(Portable, Extensible Toolkit for Scientific Computation) for handling the numerical linear algebra operations. Using Message Passing Interface (MPI) for distributed memory
helps us to improve the performance of the spatially variant code when it is executed on a parallel system. 
\end{itemize}

{\sl Fall 2018 - Postdoc Volunteer, Bioinformatics Department - UTEP} %\hfill \\      
\begin{itemize}\itemsep -2pt
\item Systemic Bioinformatic analysis requires shepherding files through a series of transformations, called a pipeline or a workflow.
These workflow data products are used to extract data and highlight relevant information that biologists examine which then validate targeted 
experiments and support a hypothesis or help to formulate a new one. These new custom bioinformatic analysis tools require programmers to implement
new methods and/or put together existing ones to build new data analysis frameworks (data flows commonly known as bioinformatic pipelines).
These transformations are done by executable command-line third-party software written for Unix/Linux-compatible operating systems. The increasing amount of DNA
sequences has intensified the need for robust workflows for interpreting a range of biological phenomena.
\end{itemize}

{\sl Fall 2017 and Spring 2018 - Oak Ridge National Laboratory \& University of Texas at El Paso} \hfill  
\begin{itemize}
\item Computer code implementation of the 3D Spatially Variant Lattice (SVL) algorithm.
The SVL written program codes were implemented on the TACC Stampede 2 supercomputer using C in conjunction with the PETSC and FFTW libraries.
We studied the scalability and performance of the SVL algorithm on the Stampede 2 architecture (Intel KNL and Skylake).
\end{itemize}

{\sl Summer Internship 2017, Oak Ridge National Laboratory - Computational Science Institute - Future Technologies Group  }% \hfill \\  
\begin{itemize}
\item Benchmarked the the High-Performance Geometric Multigrid (HPGMG) and PETSC 2D Spatially Variant Lattice (SVL) algorithm.
Used PIN for the instrumentation framework. Studied performance on ORNL ExCL HPC systems.
\end{itemize}

{\sl Summer Internship 2015, Border Biomedical Research Center Bioinformatics Department- UTEP} %\hfill \\      
\begin{itemize}\itemsep -2pt
\item Developed PYTHON and WEB-PYTHON application for DNA (Genomics and Sequence Analysis). 
Evaluated performance using HTCondor job scheduler for workload management and multiple submission jobs. 
\end{itemize}
%
{\sl Summer Internship 2014,  Electronic Structure Group - Physics Department - UTEP} %\hfill \\      
\begin{itemize}\itemsep -2pt
\item Performance evaluation and analysis of the Naval Research Laboratory Molecular Orbital Library (NRLMOL) code, Poisson subroutines in Fortran.
The NRLMOL code is a set of parallel programs developed by Mark Pederson and collaborators written in FORTRAN 77
with updated subroutines written in FORTRAN 90 and C. A performance evaluation was done with the help of PERFTOOLS, which is a performance evaluation library available at NERSC.
\end{itemize} 
%
{\sl Math Department, UNM, Albuquerque-NM}% \hfill \\
\begin{itemize}  \itemsep -2pt %reduce space between items
\item Bacterial Symbiosis and its influence on the Dynamics of Chagas Disease.
Developed a numerical approach to simulate the introduction of genetically modified bacterial symbiont into natural populations of Chagas disease vectors.
This approach utilizes the coprophagic behavior of these insects, which is the way in which the symbiont is transmitted among
bug populations in Kissing Bugs nature. The numerical simulation was implemented using  MATLAB.
\end{itemize}
%
{\sl Physics Department,UNMSM, Lima-Peru} %\hfill \\      
\begin{itemize}\itemsep -2pt
\item Study of Structural Property for Reflectivity Modeling System Superlattices.
\begin{itemize}
\item Implemented a numerical model in FORTRAN  based on the dynamical diffraction theory to model the experimental reflectivity.
\item Numerical model includes material mixing at the interface, interface roughness and random variation of component thickness.
\item The numerical model reveals the structural parameters of the device, such as the multilayer period, the individual layer thickness, the width of the interface and the optical constants. 
\item The numerical model is used to study superlattices devices composed of hydrogenated amorphous silicon/silicon carbide (a-Si:H/a- Si1 - xCx:H) and silicon/germanium (a-Si:H/a-Ge:H), deposited by the plasma-enhanced chemical vapor deposition (PECVD) technique, were
analyzed using small-angle X-ray diffraction.
\end{itemize}
\end{itemize} 
%
{\sl Geophysical Institute of Peru (IGP), Lima-Peru}% \hfill \\      
\begin{itemize}\itemsep -2pt
\item Department of Emergency-Training program and earthquake data management and earthquake effect in real time.
\begin{itemize}
\item Record daily earthquake data collection, earthquake wave propagation gives information about the earthquake location of earthquake effect in real time.
\item Estimation of earthquakes location by measuring of the wave ground response of the compressive P wave, the shearing S wave.
\item Rolling surface wave motions recorded by seismographs stations.
\end{itemize}
\end{itemize} 

\section{HONORS} 
\begin{itemize}
 \item {\sl (ATPESC) Argonne Training Program in Extreme-Scale Computing (\it {2020})} %\hfill \\
 \item {\sl SC Conference Student Volunteer (\it {2014, 2015, 2016 (SciNet), 2017})} %\hfill \\
 \item {\sl Kappa Mu Epsilon (\it {Math Honor Society})} %\hfill \\      
 \item {\sl SIAM Student Chapter vice-president 2013}
 \item {\sl SIAM Student Chapter president 2014} %\hfill \\ 
 \item {\sl UTEP SIAM Seminar coordinator (2015, 2016)}%\hfill\\
 \item {\sl Member of Student Advisory Committee for the International Student Fellowship at the First Baptist Church El Paso \bf (\it {2014})}%\hfill\\
 \item {\sl Recipient of Good Neighbor Scholarship during the years 2012-2016}   %\hfill  
\end{itemize}

\section{PUBLICATIONS}
{\sl Reflectivity modeling of Si-based amorphous superlattices; Superlattices and Microstructures.}
E. L. Zevallos Velásquez,\textbf{ H. Moncada L.}, UNMSM-Perú, M. C. A. Fantini USP-Brazil, Reflectivity modeling of Si-based amorphous superlattices;
Superlattices and Microstructures, Vol 28, No 3, 2000.

% {\sl XSEDE Conference, 2016  Miami, FL}.
% Poster Presentation,Towards the Salability and Hybrid Parallelization of a Spatally Variant Lattice Algorithm
% \textbf{Henry. R. Moncada L., Shirley V. Moore, Raymond C. Rumpf}

% {\sl Under Review: Parallelization and Scalability Analysis of the 3D Spatially Variant Lattice.}
% \textbf{Henry R. Moncada}, UTEP; Shirley V. Moore, ORNL; Raymond C. Rumpf, UTEP.
% ACME 2018 special session of the 2018 International Conference on High Performance Computing \& Simulation (HPCS 2018).
\section{REFERENCES}

{\bf Dr. Shirley V Moore}\\
(865)719-0701\\
\verb+svmoore@utep.edu+\\
Associate Professor, Computer Science\\
University of Texas at El Paso

{\bf Dr. Philip Wiley Jones} \\
(505) 667 6387\\
\verb+pwjones@lanl.gov+\\
Los Alamos National Laboratory

{\bf Dr. Paul Delgado} \\
(915)588-0876\\
\verb+dr.paul.m.delgado@gmail.com+\\
Computational Thermal Engineer,
Ball Aerospace Inc.


% {\bf Dr. Raymond C. Rumpf}\\
% (915) 747 6958\\
% \verb+rcrumpf@utep.edu+\\
% Professor, Department of Electrical and Computer Engineering, 
% Director EM Lab Schellenger,
% University of Texas at El Paso.
% 
% {\bf Dr. Ming-Ying Leung}\\
% (915) 747-6836\\%(915)227-0329 /
% \verb+mleung@utep.edu+\\
% Professor, Mathematical Sciences,
% Director, Bioinformatics and Computational Science Programs
% Director, BBRC Bioinformatics Core,
% University of Texas at El Paso.

% {\bf Cynthia Aguilar-Davis}\\
% \verb+cmaguilar3@utep.edu+\\
% (915) 747- 6407\\
% Manager Program, Computational Sciences,
% University of Texas at El Paso.
\end{resume}

\end{document}
