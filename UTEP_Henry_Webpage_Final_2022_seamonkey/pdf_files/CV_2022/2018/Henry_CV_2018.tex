% LaTeX resume using res.cls
\documentclass[margin]{res}
%\usepackage{helvetica} % uses helvetica postscript font (download helvetica.sty)
%\usepackage{newcent}   % uses new century schoolbook postscript font 
\setlength{\textwidth}{5.3in} % set width of text portion

\begin{document}
% Center the name over the entire width of resume:
 \moveleft.5\hoffset\centerline{\large\bf Henry R. Moncada}
% Draw a horizontal line the whole width of resume:
 \moveleft\hoffset\vbox{\hrule width\resumewidth height 1pt}\smallskip
% Address begins here
% Again, the address lines must be centered over entire width of resume:
 \moveleft.5\hoffset\centerline{732 Elmwood ct, Apt B}
 \moveleft.5\hoffset\centerline{El Paso, TX 79932}
 \moveleft.5\hoffset\centerline{(915) 730-2550}
 \moveleft.5\hoffset\centerline{hrmoncada@gmail.com}
% Draw a horizontal line the whole width of resume:
\moveleft\hoffset\vbox{\hrule width\resumewidth height 1pt}\smallskip
\begin{resume}
\section{OBJECTIVE} Job application% for Postdoctoral Researcher 
% \section{STATUS}
% {\sl Peruvian citizen}\\
% {\sl Student visa\textbf{ F1}}
\section{EDUCATION} % \sl will be bold italic in New Century Schoolbook (or any postscript font) and just slanted in Computer Modern (default) font
{\sl \bf Ph.D Computational Science}, \\
University of Texas at El Paso (UTEP), El Paso, TX, US, May 14, 2018, (GPA: 3.9) \ \\ 
\it``Parallelization and Scalability Analysis of the 3D Spatially Variant Lattice Algorithm''\\
%
{\sl \bf Master of Science in Computational Science,} \\
University of Texas at El Paso, El Paso, TX, US, May 14, 2016, (GPA: 3.9) \\
\it``Towards the Scalability and Hybrid Parallelization of a Spatially Variant Lattice Algorithm''\\
%
{\sl \bf Master of Science in Applied Mathematics}\\
University of New Mexico (UNM), Albuquerque, NM, US, Aug 1, 2009, (GPA: 3.83) \\
\it``Paratransgenic Vectors and their Potential Influence on the Dynamics of Chagas disease'' \\    
%
{\sl\bf  Bachelor of Science, Physics}\\
Universidad Nacional Mayor de San Marcos (UNMSM), Lima, Peru, 1998\\
\it``Study of Structural Property for Reflectivity Modeling System Superlattices''   
                 
\section{COMPUTER \\ SKILLS}
{\sl Operating Systems:} Linux, Windows%\hfill\\

{\sl High-Level language (HLL) :} %\hfill \\
  \begin{itemize}  \itemsep -1pt 
      \item Proficient in C/C++, Python 2.7/3, FORTRAN 77/90, and Matlab/Octave 
      \item Familiar with R and Maple/Maxima
      \item Familiar with MPI/OpenMP - for CPU multicore clusters
      \item Familiar with CUDA - graphical processing units (GPUs)
      \item Familiar with OpenCL/OpenACC/Trilinos-Kokkos - Heterogeneous platforms (CPUs \& GPUs)     
    \end{itemize}

{\sl Version Control Software :} %\hfill \\
  \begin{itemize}  \itemsep -1pt 
      \item GIT and Subversion (SVN) 
    \end{itemize}    
    
{\sl Concurrent and Parallel Programming Tools}% \hfill \\
  \begin{itemize}  \itemsep -1pt 
   \item Parallel Performance Evaluation
  \begin{itemize}  \itemsep -1pt  
      \item Lightweight Profiling Library for MPI (mpiP)
      \item Cray Pertools
      \item Tuning Analysis Utilities (TAU)
    \end{itemize}
   \item Computational Math Libraries
  \begin{itemize}  \itemsep -1pt  
      \item Basic Linear Algebra Subprograms (BLAS) \& Linear Algebra PACKage (LAPACK)   
      \item Fastest Fourier Transform in the West (FFTW)
      \item SuiteSparse(CSPARSE)
      \item Portable, Extensible Toolkit for Scientific Computation (PETSc)
      \item Kokkos - Trilinos Core Kernels Package.
    \end{itemize}
  \end{itemize}    
    
\section{LANGUAGES}
{\sl Spanish,} native language\\
{\sl English,} speak fluently and read and write with high proficiency

\section{RESEARCH EXPERIENCE}
{\sl Postdoc Volunteer, Bioinformatics Department - UTEP} %\hfill \\      
\begin{itemize}\itemsep -2pt
\item Systemic Bioinformatic analysis requires shepherding files through a series of transformations, called a pipeline or a workflow.
These workflow data products are used to extract data and highlight relevant information than biologists examine which then validate targeted 
experiments and support hypothesis or help to formulate a new one. These new custom bioinformatic analysis tools require programmers to implement
new methods and/or put together existing ones to build new data analysis frameworks (data flows commonly known as bioinformatic pipelines).
These transformations are done by executable command-line third-party software written for Unix/Linux-compatible operating systems. The increasing among of DNA
sequences has intensified the need for robust workflows for interpreting a range of biological phenomena.
\end{itemize}


{\sl Fall 2017 and Spring 2018 - Oak Ridge National Laboratory \& University of Texas at El Paso} \hfill %\\  
\begin{itemize}
\item I had been working on the computer code implementation of the 3D Spatially Variant Lattice (SVL) algorithm.
\begin{itemize}\itemsep -2pt
\item The SVL  written program codes were implemented on TACC Stampede 2 supercomputer.
\item The SVL code was written using C in conjunction with PETSC and FFTW tools.
\item We study the Scalability and performance of the SVL on Stampede 2 architecture (KNL and SKX).
\end{itemize}
\end{itemize}

{\sl Summer Internship 2017, Oak Ridge National Laboratory - Computational Science Institute - Future Technologies Group  }% \hfill \\  
\begin{itemize}
\item I had been working on the benchmark of the High-Performance Geometric Multigrid (HPGMG) and PETSC 2D Spatially Variant Lattice (SVL) algorithm.
\begin{itemize}\itemsep -2pt
\item The following computer tools were used,  
\item PIN for the instrumentation framework and MIC instruction-set architectures were used to benchmark a serial version of the  HPGMG and SVL code.
\item PIN will provide an API that extracts away the underlying instruction-set idiosyncrasies and allows context information such as register contents to be passed to the injected code as parameters.
\item PIN saves and restores the registers that are overwritten by the injected code, so the application continues to work. 
\item This work was develop using ORNL HPC server EXCL and NEWARK/MEGATRON supercomputer.
\end{itemize}
\end{itemize}

{\sl Summer Internship 2015, Border Biomedical Research Center Bioinformatics Department- UTEP} %\hfill \\      
\begin{itemize}\itemsep -2pt
\item I have been working on the development PYTHON and WED-PYTHON application for DNA - Genomics and Sequence Analysis.
\item A performance evaluation using HTCondor job scheduler for workload management and multiple submission jobs. 
\begin{itemize}
\item Project focus on development python bioinformatics computational tools to predict genomic structures and analyze molecular sequences
for the G protein-coupled receptors (GPCRs). GPCR constitute a large protein family of receptors that sense molecules outside the cell and activate inside signal transduction pathways and cellular responses.
\item There are an incredible amount of GPCR-DNA data available to be analyzed and used on the computational model.
\item A performance evaluation of HTCondor job scheduler system using Python and BLAST to evaluated DNA sequences. 
\end{itemize}

% HTCondor submition file containing commands that allow jobs to be submitted through Linux, Unix, Mac OS X, FreeBSD, and contemporary Windows operating systems..
% HTCondor locates a machine that can run each job within the pool of machines, packages up the job and ships it off to this execute machine.
% The jobs run, and output is returned to the machine that submitted the jobs.
\end{itemize}
%
{\sl Summer Internship 2014,  Electronic Structure Group - Physics Department - UTEP} %\hfill \\      
\begin{itemize}\itemsep -2pt
\item In the summer 2014, I was working on the performance evaluation and analysis of the
Naval Research Laboratory Molecular Orbital Library (NRLMOL) code, Poisson subroutines in Fortran.
\begin{itemize}
\item The NRLMOL code is a set of parallel programs developed by Mark Pederson and collaborators written in FORTRAN 77
with updated subroutines written in FORTRAN 90 and C.
\item It is used on the calculation of electronic structure for large molecules and clusters, group symmetry and PBE exchange-correlation functional, etc. 
\item NRLMOL is an existing massively parallel code based on multicore processors that can reach a very good flops performance which I compiled and executed using NERSC-HOPPER petaflop system on Cray XE6 supercomputer.
\item The goal of the internship was to develop an understanding of NRLMOL code on a high-performance platform, and moreover identify possible bottlenecks and give suggestions for their improvement. 
\item A performance evaluation was done with the help of PERFTOOLS, which is a performance evaluation library available at NERSC.
PERFTOOLS is used to profile the NRLMOL code. 
\item The profile information is used to identify possible performance bottlenecks, load imbalance and tracing as well as to determine the possible locations and causes of those bottlenecks.
\item The results given by PERFTOOLS about the NRLMOL code can be used to improve the performance of the NRLMOL and develop a potential and efficient simulation for a particular HPC platform.
\end{itemize}

\end{itemize} 
%
{\sl Math Department, UNM, Albuquerque-NM}% \hfill \\
\begin{itemize}  \itemsep -2pt %reduce space between items
\item Bacterial Symbiosis and its influence on the Dynamics of Chagas Disease.
\begin{itemize}
\item Developed a numerical approach to simulate the introduction of genetically modified bacterial symbiont into natural populations of Chagas disease vectors.
\item This approach utilizes the coprophagic behavior of these insects, which is the way in which the symbiont is transmitted among
bug populations in Kissing Bugs nature.
\item Implement the numerical simulation in MATLAB
\end{itemize}
\end{itemize}
%
{\sl Physics Department,UNMSM, Lima-Peru} %\hfill \\      
\begin{itemize}\itemsep -2pt
\item Study of Structural Property for Reflectivity Modeling System Superlattices.
\begin{itemize}
\item Implement a numerical model in FORTRAN  based on the dynamical diffraction theory to model the experimental reflectivity.
\item Numerical model includes material mixing at the interface, interface roughness and random variation of component thickness.
\item The numerical model reveals the structural parameters of the device, such as the multilayer period, the individual layer thickness, the width of the interface and the optical constants. 
\item The numerical model is used to study superlattices devices composed of hydrogenated amorphous silicon/silicon carbide (a-Si:H/a- Si1 − xCx:H) and silicon/germanium (a-Si:H/a-Ge:H), deposited by the plasma-enhanced chemical vapor deposition (PECVD) technique, were
analyzed using small-angle X-ray diffraction.
\end{itemize}
\end{itemize} 
%
{\sl Geophysical Institute of Peru (IGP), Lima-Peru}% \hfill \\      
\begin{itemize}\itemsep -2pt
\item Department of Emergency-Training program and earthquake data management and earthquake effect in real time.
\begin{itemize}
\item Record daily earthquake data collection, earthquake wave propagation gives information about the earthquake location of earthquake effect in real time.
\item Estimation of earthquakes location by measuring of the wave ground response of the compressive P wave, the shearing S wave.
\item Rolling surface wave motions recorded by seismographs stations.
\end{itemize}
\end{itemize} 

% \section{RESEARCH INTEREST}
% I am interested in using mathematical models and High-Performance Computing (HPC) power to understand and simulate problems arising from science and engineering. My goal is to combine parallel tools like OpenMP, MPI and CUDA/OpenCL libraries to perform complex calculations so that we can achieve a supercomputer's
% performance level without too much cost in calculation time and overhead.
% 
% I am particular interested in the following  groups areas:
% \begin{enumerate}
% \item Math and Science relative implementation
% \begin{itemize}
% %\item Optimization
% \item Mathematical Modeling %Algorithms for solution of differential and integral equations
% %\item Parallel Computing
% %\item Multiscale methods
% %\item Bioinformatics
% \item Machine learning
% \item Network Security
% \end{itemize}
% \item Parallel Computing implementation
% \begin{itemize}
% \item HPC cluster management
% \item Scalable cluster computer and heterogeneous computing
% %\item Scalable solvers
% %\item Environment for scalable computing
% %\item Paralle I/O
% \item Data intencive computation
% \item 3D Visualization
% %\item FPGA
% \end{itemize}\end{enumerate}
% 
% There are many areas in which HPC in combination with numerical techniques can be used. Below, I am pointing out few ones that I am particularly
% interested in developing.
% 
% \begin{itemize}\itemsep -2pt
% \item Biology: Using mathematical models to understand the biological processes that shape population and community dynamics, with a
% particular interest in the evolution of infectious diseases and modeling problems in systems biology. 
% \item Geoscience: Use mathematical models to study the dynamics and elasticity of plate tectonic processes that
% cause  earthquakes and volcanism with their destructive consequences including earth materials. 
% Use of remote sensing and digital data analysis to study the dynamic interactions between climate, tectonics, and surface processes,
% dynamics governing abrupt changes in the atmosphere-ocean system.
% \item Physics: Use numerical analyses % algorithm design t o study their implementation
% to study the dynamics of material mixing, molecular dynamics, elasticity, material transport, radiation transport and multi-scale material modeling.  
% \item Electrodynamics Applications : Use HPC as well as mathematical tools to impletement and speed up computational simulation eletrodynamics applications. 
% \end{itemize}
% 
% In the course of my research, I have noticed a number of hardware advances in parallel computing architecture, insights into algorithms as well as analysis techniques aid in the design of elegant and simple solutions.
% I envisage the field of mathematical modeling created from the ground up, building upon the foundations of a number of fields. One in particular field in computer science get my attention, machine learning.
% Even though, I did not work directly with machine learning. Many of the mathematical armories I am awarded. It has been used in machine learning.   
% 
% % In the near future
% In the near future, I am interested in the principles involved in the design of necessary application of machine learning in combination with frameworks of mathematical modeling and parallel computing.
% Parallelism can play a crucial role in improving the performance of machine learning techniques. Since machine learning uses methods of data analysis that automates analytical 
% models technique that teaches computers to do what comes naturally to humans and animals;
% learn from experience and improve their learning over time in autonomous fashion, by feeding them data and information in the form of observations and real-world interactions.
% Machine learning requires the use of intensive data analysis to improve the quality of their learning over time and parallel computing can play a fundamental key in speed up
% the computational process improving the performance of machine learning models.
% The flexibility of machine learning techniques models allows researchers, data scientists, engineers, and analysts to produce reliable, repeatable decisions and results and uncover 
% hidden insights from complexity relationships and trends in the data and mathematical model  techniques. %\cite{Tarca_2007_Machine_Learning_Biology}.
\section{DOCTORAL RESEARCH TOPIC}
% 3D SVL
The purpose of this research is to design a faster implementation of the spatially variant that improves its performance when it is running on a parallel computer system. 
The spatially variant is used to synthesize a spatially variant lattice for a periodic electromagnetic structure. The spatially variant has the ability to spatially vary the unit
cell orientation and exploit its directional dependencies. The spatially variant produces a lattice that is smooth, continuous and free of defects. The lattice spacing remains strikingly
uniform when the unit cell orientation, lattice spacing, fill fraction and more are spatially varied. This is important for maintaining consistent properties throughout the lattice.
Periodic structures like a photonic crystal or metamaterial devices can be enhanced using the spatially variant to unlock new physics applications.
Our current effort is to write a portable spatially variant code for parallel architectures. To develop and write the code, we pick a general-purpose programming language
that supports structured programming. For the parallel code, we use FFTW for handling the Fourier Transform of the unit cell device and PETSc 
(Portable, Extensible Toolkit for Scientific Computation) for handling the numerical linear algebra operations. Using Message Passing Interface (MPI) for distributed memory
helps us to improve the performance of the spatially variant code when it is executed on a parallel system. 

% % 2D SVL, Poisson subroutines in Fortran.
% My current project consists of the parallelisation of a 2D periodic structure of metamaterial. The artificially
% fabricated materials are designed to control, direct, and manipulate sound waves as these might occur in gases, liquids, and solids.
% Our final goal is to complete the simulations for a 3D lattice structure.
% The computation of this project requires heavy computing calculation that can be accelerated using HPC as a major tool. We propose a parallel
% programming approach using CUDA, OpenMP, and MPI programming. The use of these libraries as well as the mixed combination of these codes
% could potentially offer the most effective parallelisation strategy. 


% \section{TEACHING EXPERIENCE} 
% {\sl Colleges} \hfill \\
%   \\
% \textbf{Instructor (Full Time)}, Universidad Nacional del Callao (UNAC), Callao-Peru, Faculty of Natural Science and Mathematics. \hfill { \it 2010 to July-2012\/} \\
% \textbf{UNM-Mathematics Department}, Teacher assistant (TA).  \hfill {\it Spring 2006 to 2009 \/}\\
% \textbf{UNM-Physics Department}, Part-time teacher assistant (PTA). \hfill {\it Fall 2007 to Spring 2008\/}\\
% \textbf{UNM-Multicultural Engineering Program}, Residential assistant (RA). \hfill {\it Summer 2007, 2008\/}\\
% \textbf{UNM-Johnson Gym}, Math-Tutor. \hfill {\it Fall 2006\/} \\
% \textbf{Instructor (Part Time)}, Universidad Nacional Mayor de San Marcos (UNMSM), Lima-Peru, Faculty of Physics \hfill {\it April-December 2002, April-July 2003\/} \\
% \textbf{Instructor (Full Time)}, Universidad Nacional del Callao (UNAC), Callao-Peru, Faculty of Natural Science and Mathematics.  \hspace{3cm}    {\it April-1998 to July-2003\/}    
% \begin{itemize}  \itemsep -2pt %reduce space between items
% \item Mathematics and Physics Instructor: Provide instructional and support for math and physics classes. Activities lecture/discussion/problem/lesson-review/ supervision of labs-activities that augmented the basic first-year courses, grade papers,
% quizzes, and exams. Also, I assisted students in the core mathematics
% and physics classes. For some of the courses, I was responsible for writing my syllabus, as well as choose (or create) the homework problems, worksheets, and labs. 
% \end{itemize}
% %\clearpage
\section{SKILLS}  
{\sl Problem Solving} \hfill \\ 
   \begin{itemize}\itemsep -2pt
      \item Excellent analytical and logical reasoning skills.  Able to multi-task. Can learn new skills quickly. Able to lead or
            work within a group environment.
  \end{itemize} 

{\sl Teaching skill} \hfill \\      
    \begin{itemize}\itemsep -2pt
      \item Enjoy working and helping students to increase their motivation to study math and physics 
            and develop and improve new learning skills to solve math and physics problems.
    \end{itemize} 

\section{HONORS} 
\begin{itemize}
 \item {\sl Kappa Mu Epsilon (\it {Math Honor Society})} %\hfill \\      
 \item {\sl SIAM Student Chapter vice-president 2013}
 \item {\sl SIAM Student Chapter president 2014} %\hfill \\ 
 \item {\sl UTEP SIAM Seminar coordinator (\it {2015, 2016})}%\hfill\\
 \item {\sl Member of Student Advisory Committee for the International Student Fellowship at the First Baptist Church El Paso} \bf (\it {2014})%\hfill\\
 \item {\sl Recipient of Good Neighbor Scholarship during the following year period 2012-2016}   %\hfill  
\end{itemize}

\section{PUBLICATIONS}
{\sl Reflectivity modeling of Si-based amorphous superlattices; Superlattices and Microstructures.}
E. L. Zevallos Velásquez,\textbf{ H. Moncada L.}, UNMSM-Perú, M. C. A. Fantini USP-Brazil, Reflectivity modeling of Si-based amorphous superlattices;
Superlattices and Microstructures, Vol 28, No 3, 2000.

{\sl XSEDE Conference, 2016  Miami, FL}.
Poster Presentation,Towards the S
alability and Hybrid Parallelization of a Spatally Variant Latti
e Algorithm
\textbf{Henry. R. Moncada L., Shirley V. Moore, Raymond C. Rumpf}

{\sl Under Review: Parallelization and Scalability Analysis of the 3D Spatially Variant Lattice.}
\textbf{Henry R. Moncada}, UTEP; Shirley V. Moore, ORNL; Raymond C. Rumpf, UTEP.
%ACME 2018 special session of the 2018 International Conference on High Performance Computing \& Simulation (HPCS 2018).
\newpage
\section{REFERENCES}
{\bf Dr. Paul Delgado} \\
(915)588-0876\\
\verb+dr.paul.m.delgado@gmail.com+\\
Computational Thermal Engineer,
Ball Aerospace Inc.

{\bf Dr. Shirley V Moore}\\
(865)719-0701\\
\verb+mooresv@ornl.gov+\\
Senior Research,
Computer Science and Mathematics Division,
Oak Ridge National Laboratory.

{\bf Dr. Raymond C. Rumpf}\\
(915) 747 6958\\
\verb+rcrumpf@utep.edu+\\
Professor, Department of Electrical and Computer Engineering, 
Director EM Lab Schellenger,
University of Texas at El Paso.

{\bf Dr. Ming-Ying Leung}\\
(915) 747-6836\\%(915)227-0329 /
\verb+mleung@utep.edu+\\
Professor, Mathematical Sciences,
Director, Bioinformatics and Computational Science Programs
Director, BBRC Bioinformatics Core,
University of Texas at El Paso.


% {\bf Cynthia Aguilar-Davis}\\
% \verb+cmaguilar3@utep.edu+\\
% (915) 747- 6407\\
% Manager Program, Computational Sciences,
% University of Texas at El Paso.
\end{resume}
\end{document}