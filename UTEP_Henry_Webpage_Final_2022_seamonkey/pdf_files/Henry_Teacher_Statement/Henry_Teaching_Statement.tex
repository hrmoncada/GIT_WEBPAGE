%\documentstyle[11pt,a4]{article}
%\documentclass[a4paper]{article}
\documentclass[a4paper, 11pt]{article}
% Seems like it does not support 9pt and less. Anyways I should stick to 10pt.
%\documentclass[a4paper, 9pt]{article}
\topmargin-2.0cm

\usepackage{fancyhdr}
\usepackage{pagecounting}
\usepackage[dvips]{color}
%%%%%%%%%%%%%%%%%%%%%%%%%%%%%%%%%%%%%%%%%%%%%%%%%%%%%%
% Math
%%%%%%%%%%%%%%%%%%%%%%%%%%%%%%%%%%%%%%%%%%%%%%%%%%%%%%
\usepackage{amsmath}
\usepackage{amssymb}
\usepackage{amsfonts} %para usar as fontes %matematicas, e pega fontes para \mathbb
\usepackage{amsthm}   %ambientes com teoremas (usar depois de amsmath)
%\usepackage{biblatex}
%\usepackage{natbib}
% Color Information from - http://www-h.eng.cam.ac.uk/help/tpl/textprocessing/latex_advanced/node13.html

% NEW COMMAND
% marginsize{left}{right}{top}{bottom}:
%\marginsize{3cm}{2cm}{1cm}{1cm}
%\marginsize{0.85in}{0.85in}{0.625in}{0.625in}

\advance\oddsidemargin-0.65in
%\advance\evensidemargin-1.5cm
\textheight9.2in
\textwidth6.75in
\newcommand\bb[1]{\mbox{\em #1}}
\def\baselinestretch{1.05}
%\pagestyle{empty}

\newcommand{\hsp}{\hspace*{\parindent}}
\definecolor{gray}{rgb}{0.4,0.4,0.4}
%\definecolor{gray}{rgb}{1.0,1.0,1.0}

% %%%%%%%%%%%%%%%%%%%%%%%%%%%%%%%%%%%%%%%%%%%%%%%%%%%%%%
% % Graphs path to the Picture Folder
% %%%%%%%%%%%%%%%%%%%%%%%%%%%%%%%%%%%%%%%%%%%%%%%%%%%%%%
% %\graphicspath{/home/henry/Desktop/UTEP_Proposal_Thesis/PhD_Thesis/Pictures}
% \graphicspath{{Pictures/}{Data/}{References}} % Two folders Picture and Data    
% %%%%%%%%%%%%%%%%%%%%%%%%%%%%%%%%%%%%%%%%%%%%%%%%%%%%%%
% % page edition
% %%%%%%%%%%%%%%%%%%%%%%%%%%%%%%%%%%%%%%%%%%%%%%%%%%%%%%
% %\linespread{2.5}
% \addtolength{\textwidth}{3cm}
% \addtolength{\textheight}{3cm}
% \addtolength{\hoffset}{-2cm}
% % In case you need to adjust margins:
% \topmargin=-0.5 in      %
% \evensidemargin= .5 in     %
% \oddsidemargin= .5 in      %
% \textwidth = 7 in        %
% \textheight = 9.0 in       %
% \headsep = 0.15 in  
% \pagestyle{fancyplain}
% \lhead{\fancyplain{}{\thepage/\totalpages{}}}}
% \rhead{\fancyplain{}{Henry R Moncada}}


%%%%%%%%%%%%%%%%%%%%%%%%%%%%%%%%%%%%%%%%%%%%%%%%%%%%%%
% bibliography
%%%%%%%%%%%%%%%%%%%%%%%%%%%%%%%%%%%%%%%%%%%%%%%%%%%%%%
%\usepackage{biblatex}
\bibliographystyle{plain}
%\bibliography{urlbib}

\begin{document}
\thispagestyle{fancy}
%\pagenumbering{gobble}
%\fancyhead[location]{text} 
% Leave Left and Right Header empty.
\lhead{}
\rhead{}
%\rhead{\thepage}
\renewcommand{\headrulewidth}{0pt} 
\renewcommand{\footrulewidth}{0pt} 
\fancyfoot[C]{\footnotesize \textcolor{gray}{ }} 

%\pagestyle{myheadings}
%\markboth{Sundar Iyer}{Sundar Iyer}

\pagestyle{fancy}
\lhead{\textcolor{gray}{\it Henry R. Moncada}}
\rhead{\textcolor{gray}{\thepage/\totalpages{}}}
%\rhead{\thepage}
%\renewcommand{\headrulewidth}{0pt} 
%\renewcommand{\footrulewidth}{0pt} 
%\fancyfoot[C]{\footnotesize http://www.stanford.edu/$\sim$sundaes/application} 
%\ref{TotPages}

% This kind of makes 10pt to 9 pt.
\begin{small}

%\vspace*{0.1cm}
\begin{center}
{\LARGE \bf TEACHING STATEMENT}\\
\vspace*{0.1cm}
{\normalsize Henry R. Moncada (hrmoncada@gmail.com)}
\end{center}
%\vspace*{0.2cm}

% Write about research interests...
%\footnotemark
%\footnotetext{Check This}

%https://www.math.uconn.edu/taprogram/teaching-statement-guidelines-and-samples/

During my time as a graduate student at UTEP at UNM on which I worked as a research assistant and teacher assistant.
I discover several fundamental tenets of teaching and learning that I am continually rediscovering:
\begin{itemize}
\item Mathematics is not a subject that many people find entertaining to practice.
\item Mathematics is an incredibly subject if proper motivation and explanation are used to facilitate the learning process.
\item Mathematics is a language apart from any other language (such as English/Spanish/etc.) and a word that many students do not speak.
\end{itemize}
These tenets shape my philosophy of teaching and learning, as well as influence my actions as a teacher and student, inside and outside of the classroom. 
I started my teaching experience just after I public defended my bachelor degree thesis project in physics. 
I began in the Faculty of Natural Science and Mathematics at the ``\textit{Universidad Nacional del Callao}`` a government university on the most import and biggest port of Peru and Sud-American Pacific coast. 
On which I remain until Aug 2003. 

In 2001,  I started working at the ``\textit{Universidad Nacional Mayor de San Marcos}'' the oldest and most prominent school in my country.
I start in a part-time position giving lectures and running labs of physics courses for students in science and engineering.
I taught lectures and prepared laboratory sessions for the two first years courses of science students in mathematics and physics.
Plus, I gave workshops tutorials on solving problems in physics and math.
Also, software workshops tutorial on different programming languages such as Matlab, Fortran, C/C++, and \LaTeX\;  so students can improve their programming skills.
To evaluation student progress and retention of the material I lecture I prepare exams, assignments, labs and benchmarks metrics to score these items as fairly as possible.

I like to have an interactive classroom, whether that interaction is between the students and me, or between the students themselves. 
I believe that lecturing should not be a one-way (instructor-student).  It should be more of a dialogue between the instructor and the student.
I think it is important that the students are regularly part of the process of their own learning process. 
They need to be thinking; they need to be talking about what they are thinking.
That is why I try to make my classroom an environment where all the students feel safe to venture a response,
or make a comment, or ask a question, no matter how stupid or trivial or wrong it sounds to me or any of the other students. 
While I lecture I am constantly posing questions to the students, and the intention is that they respond to those questions,
or at the very least think deeply about them. I believe that the key to being a good teacher of mathematics
and physics is to get the students involved in their own education and get them excited about it.

In 2006, I decided to continue with my education. I was accepted into the master program of applied mathematics in the University of New Mexico.
This gave me the opportunity to teach mathematical undergraduate courses in English for the first time as a teacher assistant. 
After I finished my master degree in applied mathematics. I got back to my country and working again at the University of Callao.
Where I give lectures on numerical methods and numerical analysis. 

In 2012, I got back to the US to get a Ph.D. this time in the join field of computational science.
On May 15, 2018, I got my doctoral degree in Computational Science from the University of Texas at El Paso.
During my time at UTEP. I work as a teacher assistant and research assistant.
I was TA for the graduate core course classes of computational science for two years. I was the vice president and president of SIAM chapter.
The SIAM was developing with the idea of spread basic knowledge into student UTEP community.
As a student organization, we host seminars give by invite prominent students who shows excellent knowledge of particular topic and tools that can help to improve the particular skill of the student community.

During my time as a graduetd student I have a opportunity improve and adapt my teaching skill the US standards. For example,    
I like to give group quizzes. I have debated with myself about the educational merits of group quizzes versus individual quizzes and,
of course, there are arguments for each one.  The students ultimately need to be able to do the work on their own.
But there is a positive dynamic that comes into play when students work in groups.
It seems to me that the students learn more from talking about the problems with one another and explaining to each other exactly where
they have gone wrong if they have made a mistake. It is sometimes easier to relay what you are thinking to a small group of your peers, rather than to the whole class.
Of course, each student hands in her or his own paper for group quizzes.
I believe that quizzes are a learning tool, not just meant for testing the students. 
The grades the students receive on the quizzes show them how their methods of studying are working.
If they receive a poor grade on a quiz, perhaps it was because they have a different learning style, and they should try to study differently.

When it comes to the exams, I usually give a practice exam that I have made before and run a review session.
I think that sitting down and looking at problems that have the same difficulty as the exam is helpful for the students.
It takes the pressure off somewhat. Of course, they need to practice with good for studying resources such as old exams, quizzes, and homework, as well as the textbook.
I always try to make the exams fair not overly hard, and not overly easy.

I like to see my students succeed, and I am sad if they do not do well.
After each exam, I write the distribution of the scores on the board, so each student can get an idea of where she or he stands within the class.
For the first exam, if they are toward the bottom, I encourage them to come to my office hours to talk about it.
I ask them how they are studying, what they have been doing to learn the material.
Then I offer some suggestions on how to change their study habits in order to improve their performance. 
Perhaps they are having a stressful semester, and it is just too much.
More often than not, however, it is just a matter of trying something different and putting in that extra effort.

I am in favor of the use of technology in the classroom only as long as the technology does not interfere with the learning process.
I think that technology can be incredibly useful at times. One example would be using clickers in large lecture halls to elicit
responses from students to various questions, to make sure they have a solid grasp of the major concepts. 

I am excited at the possibility of leading undergraduate students in research. I love the area of machine learning, mathematical modeling, and computer science.
There are many projects accessible to undergraduates in a range of majors from math itself to computer science. 
I am excited to teach diverse classroom and every math, computer science class that I can! I enjoy teaching things.
I also don’t mind teaching the same thing multiple times, because it gives me the opportunity to change my teaching style and methods to see what really works.
After all, teaching is an art, and as artists teachers must always try new things in a constant effort to create a masterpiece.

I believe that universities have a significant social responsibility of
mentoring students to have open minds and rational thinking. 
Teaching gives me the opportunity to understand, learn and mold young minds to the more significant cause the betterment of both science and humanity that I passionately believe.

\section*{TEACHING EXPERIENCE} 
\textbf{Instructor (Full Time)}, Universidad Nacional del Callao (UNAC), Callao-Peru, Faculty of Natural Science and Mathematics. \hfill { \it 2010 to July-2012\/} \\
\textbf{UNM-Mathematics Department}, Teacher assistant (TA).  \hfill {\it Spring 2006 to 2009 \/}\\
\textbf{UNM-Physics Department}, Part-time teacher assistant (PTA). \hfill {\it Fall 2007 to Spring 2008\/}\\
\textbf{UNM-Multicultural Engineering Program}, Residential assistant (RA). \hfill {\it Summer 2007, 2008\/}\\
\textbf{UNM-Johnson Gym}, Math-Tutor. \hfill {\it Fall 2006\/} \\
\textbf{Instructor (Part Time)}, Universidad Nacional Mayor de San Marcos (UNMSM), Lima-Peru, Faculty of Physics \hfill {\it April-December 2002, April-July 2003\/} \\
\textbf{Instructor (Full Time)}, Universidad Nacional del Callao (UNAC), Callao-Peru, Faculty of Natural Science and Mathematics.  {\it April-1998 to July-2003\/} \\ 

\textbf{Instructor Duties and Responsibilities}
\begin{itemize}  \itemsep -2pt %reduce space between items
\item Mathematics and Physics Instructor: Provide instructional and support for math and physics classes. Activities lecture/discussion/problem/lesson-review/ supervision of labs-activities that augmented the basic first-year courses, grade papers,
quizzes, and exams. Also, I assisted students in the core mathematics
and physics classes. For some of the courses, I was responsible for writing my syllabus, as well as choose (or create) the homework problems, worksheets, and labs. 
\end{itemize}
% %\clearpage

% \vspace{0.5cm}
% %\begin{flushright}
% %Sundar Iyer
% %\end{flushright}
% 
\end{small}

%\newpage
% Change font size?
% \tiny, \footnotesize, \small,\normalsize, \large, \Large, \LARGE, and \huge 

\begin{footnotesize}
\end{footnotesize}

\end{document}

